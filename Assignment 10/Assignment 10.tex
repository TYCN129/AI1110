%%%%%%%%%%%%%%%%%%%%%%%%%%%%%%%%%%%%%%%%%%%%%%%%%%%%%%%%%%%%%%%
%
% Welcome to Overleaf --- just edit your LaTeX on the left,
% and we'll compile it for you on the right. If you open the
% 'Share' menu, you can invite other users to edit at the same
% time. See www.overleaf.com/learn for more info. Enjoy!
%
%%%%%%%%%%%%%%%%%%%%%%%%%%%%%%%%%%%%%%%%%%%%%%%%%%%%%%%%%%%%%%%

% Inbuilt themes in beamer
\documentclass{beamer}

%packages:
% \usepackage{tfrupee}
% \usepackage{amsmath}
% \usepackage{amssymb}
% \usepackage{gensymb}
% \usepackage{txfonts}

% \def\inputGnumericTable{}

% \usepackage[latin1]{inputenc}                                 
% \usepackage{color}                                            
% \usepackage{array}                                            
% \usepackage{longtable}                                        
% \usepackage{calc}                                             
% \usepackage{multirow}                                         
% \usepackage{hhline}                                           
% \usepackage{ifthen}
% \usepackage{caption} 
% \captionsetup[table]{skip=3pt}  
% \providecommand{\pr}[1]{\ensuremath{\Pr\left(#1\right)}}
% \providecommand{\cbrak}[1]{\ensuremath{\left\{#1\right\}}}
% %\renewcommand{\thefigure}{\arabic{table}}
% \renewcommand{\thetable}{\arabic{table}}      

\setbeamertemplate{caption}[numbered]{}

\usepackage{enumitem}
\usepackage{tfrupee}
\usepackage{amsmath}
\usepackage{amssymb}
\usepackage{gensymb}
\usepackage{graphicx}
\usepackage{txfonts}

\def\inputGnumericTable{}

\usepackage[latin1]{inputenc}                                 
\usepackage{color}                                            
\usepackage{array}                                            
\usepackage{longtable}                                        
\usepackage{calc}                                             
\usepackage{multirow}                                         
\usepackage{hhline}                                           
\usepackage{ifthen}
\usepackage{caption} 
\captionsetup[table]{skip=3pt}  
\providecommand{\pr}[1]{\ensuremath{\Pr\left(#1\right)}}
\providecommand{\cbrak}[1]{\ensuremath{\left\{#1\right\}}}
\renewcommand{\thefigure}{\arabic{table}}
\renewcommand{\thetable}{\arabic{table}}   
\providecommand{\brak}[1]{\ensuremath{\left(#1\right)}}
\providecommand{\brak}[1]{\ensuremath{\left(#1\right)}}

% Theme choice:
\usetheme{CambridgeUS}

% Title page details: 
\title{Assignment 10} 
\author{Vedant Bhandare (cs21btech11007)}
\date{June 2022}
\logo{\large \LaTeX{}}

\begin{document}

% Title page frame
\begin{frame}
    \titlepage 
\end{frame}

% Remove logo from the next slides
\logo{}


% Outline frame
\begin{frame}{Outline}
    \tableofcontents
\end{frame}

% Lists frame
\section{Question}
\begin{frame}{Question}
    Show that central limit theorem does not hold if the random variables $x_i$ have a Cauchy density.
\end{frame}

\section{Central Limit Theorem}
\begin{frame}{Central Limit Theorem}
    \begin{block}{Definition}
    If $X_1, X_2, X_3, ...., X_n$ are random samples drawn from a population with overall mean $\mu$ and finite variance $\sigma^2$, and $\Bar{X_n}$ is sample mean of first $n$ samples, then the limiting form of the distribution, ${\textstyle Z=\lim _{n\to \infty }{\left({\frac {{\bar {X}}_{n}-\mu }{\frac {\sigma }{\sqrt {n}}}}\right)}}$, is a standard normal distribution. 
    \end{block}    
    In other words, the central limit theorem (CLT) states that the distribution of sample means approximates a normal distribution as the sample size gets larger, regardless of the population's distribution.
\end{frame}

\begin{frame}{Conditions for CLT}
    In order for Central Limit Theorem to hold good, the distribution needs to have finite mean ($\mu$) and finite variance ($\sigma^2$).
\end{frame}

\begin{frame}{Cauchy distribution}
    Let the random variable $X_i$ have a Cauchy density
    \begin{align}
        f_{X_i}(x) = \frac{c_i}{\pi (c_i^2 + x^2)}
    \end{align}
    where, $c_i$ is the parameter.
\end{frame}
\begin{frame}{Variance}
\begin{align}
    \text{Variance} = \sigma^2 &= \int_{-\infty}^\infty x^2.f_{X_i}(x) dx\\
    \sigma^2 &= \frac{c_i}{\pi} \int_{-\infty}^\infty \frac{x^2}{c_i^2 + x^2}dx\\
    \sigma^2 &= \frac{c_i}{\pi} \int_{-\infty}^\infty \brak{1 - \frac{c_i^2}{c_i^2 + x^2}}dx\\
    \sigma^2 &= \infty
\end{align}
Thus, variance of a Cauchy distribution is infinity, which is why, Central Limit Theorem does not hold good.
\end{frame}
\end{document}
