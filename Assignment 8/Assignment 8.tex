% Inbuilt themes in beamer
\documentclass{beamer}

%packages:
% \usepackage{tfrupee}
% \usepackage{amsmath}
% \usepackage{amssymb}
% \usepackage{gensymb}
% \usepackage{txfonts}

% \def\inputGnumericTable{}

% \usepackage[latin1]{inputenc}                                 
% \usepackage{color}                                            
% \usepackage{array}                                            
% \usepackage{longtable}                                        
% \usepackage{calc}                                             
% \usepackage{multirow}                                         
% \usepackage{hhline}                                           
% \usepackage{ifthen}
% \usepackage{caption} 
% \captionsetup[table]{skip=3pt}  
% \providecommand{\pr}[1]{\ensuremath{\Pr\left(#1\right)}}
% \providecommand{\cbrak}[1]{\ensuremath{\left\{#1\right\}}}
% %\renewcommand{\thefigure}{\arabic{table}}
% \renewcommand{\thetable}{\arabic{table}}      

\setbeamertemplate{caption}[numbered]{}

\usepackage{enumitem}
\usepackage{tfrupee}
\usepackage{amsmath}
\usepackage{amssymb}
\usepackage{gensymb}
\usepackage{graphicx}
\usepackage{txfonts}

\def\inputGnumericTable{}

\usepackage[latin1]{inputenc}                                 
\usepackage{color}                                            
\usepackage{array}                                            
\usepackage{longtable}                                        
\usepackage{calc}                                             
\usepackage{multirow}                                         
\usepackage{hhline}                                           
\usepackage{ifthen}
\usepackage{caption} 
\captionsetup[table]{skip=3pt}  
\providecommand{\pr}[1]{\ensuremath{\Pr\left(#1\right)}}
\providecommand{\cbrak}[1]{\ensuremath{\left\{#1\right\}}}
\renewcommand{\thefigure}{\arabic{table}}
\renewcommand{\thetable}{\arabic{table}}   
\providecommand{\brak}[1]{\ensuremath{\left(#1\right)}}
\providecommand{\brak}[1]{\ensuremath{\left(#1\right)}}
% Inbuilt themes in beamer
\documentclass{beamer}

% Theme choice:
\usetheme{CambridgeUS}

% Title page details: 
\title{Assignment 8 \\ Papoulis Chapter 2 Question 18} 
\author{Vedant Bhandare (cs21btech11007)}
\date{May 2022}
\logo{\large \LaTeX{}}

\begin{document}

% Title page frame
\begin{frame}
    \titlepage 
\end{frame}

% Remove logo from the next slides
\logo{}


% Outline frame
\begin{frame}{Outline}
    \tableofcontents
\end{frame}

% Lists frame
\section{Question}
\begin{frame}{Question}
Ten passengers get into a train that  has three cars. Assuming a random placement of passengers, what is the probability that the first car will contain three of them?
\end{frame}

% Blocks frame
\section{Random Variable Definition}
\begin{frame}{Random Variable Definition}
\small{    
    \begin{table}[ht!]
    \centering
    \input{Table1}
    \caption{Random Variables}
    \label{table:table1}
    \end{table}
}
\end{frame}

\section{Solution}
\begin{frame}{Solution}
\begin{block}{Binomial Distribution}
\begin{align}
    Pr(X = i) = {10 \choose i} \times p^i \times (1 - p)^{10 - i}
\end{align}
\end{block}
where $i$ denotes the number of people in Car 1. The values for $i$ can be substituted in the above formula, and the graph of the PMF can be obtained.\\
\begin{align}
 p \text{ (probability of people choosing Car 1)} = \frac{1}{3}
 \end{align}
\end{frame}

\begin{frame}{Solution}
    For three people in Car 1,\\
    \begin{block}{$Pr(X = 3)$}
    \begin{align}
        Pr(X = 3) &= {10 \choose 3} \times \brak{\frac{1}{3}}^3 \times \brak{\frac{2}{3}}^7\\
        Pr(X = 3) &= 0.26
    \end{align}
    \end{block}
\end{frame}

\section{Graphs}
\begin{frame}{PMF Graph}
    \begin{figure}[!ht]
		\centering
		\includegraphics[width=\textwidth,height=5.5cm,keepaspectratio]{PMF.png}
		\caption{Probability Mass Function}
		\label{fig1}
	\end{figure}
\end{frame}

\end{document}
