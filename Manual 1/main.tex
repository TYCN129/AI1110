\documentclass[journal,12pt,twocolumn]{IEEEtran}
\usepackage{setspace}
\usepackage{gensymb}
\usepackage{caption}
%\usepackage{multirow}
%\usepackage{multicolumn}
%\usepackage{subcaption}
%\doublespacing
\singlespacing
\usepackage{csvsimple}
\usepackage{amsmath}
\usepackage{float}
\usepackage{multicol}
%\usepackage{enumerate}
\usepackage{amssymb}
%\usepackage{graphicx}
\usepackage{newfloat}
%\usepackage{syntax}
\usepackage{listings}
\usepackage{color}
\usepackage{tikz}
\usetikzlibrary{shapes,arrows}



%\usepackage{graphicx}
%\usepackage{amssymb}
%\usepackage{relsize}
%\usepackage[cmex10]{amsmath}
%\usepackage{mathtools}
%\usepackage{amsthm}
%\interdisplaylinepenalty=2500
%\savesymbol{iint}
%\usepackage{txfonts}
%\restoresymbol{TXF}{iint}
%\usepackage{wasysym}
\usepackage{amsthm}
\usepackage{mathrsfs}
\usepackage{txfonts}
\usepackage{stfloats}
\usepackage{cite}
\usepackage{cases}
\usepackage{mathtools}
\usepackage{caption}
\usepackage{enumerate}	
\usepackage{enumitem}
\usepackage{amsmath}
%\usepackage{xtab}
\usepackage{longtable}
\usepackage{multirow}
%\usepackage{algorithm}
%\usepackage{algpseudocode}
\usepackage{enumitem}
\usepackage{mathtools}
\usepackage{hyperref}
%\usepackage[framemethod=tikz]{mdframed}
\usepackage{listings}
    %\usepackage[latin1]{inputenc}                                 %%
    \usepackage{color}                                            %%
    \usepackage{array}                                            %%
    \usepackage{longtable}                                        %%
    \usepackage{calc}                                             %%
    \usepackage{multirow}                                         %%
    \usepackage{hhline}                                           %%
    \usepackage{ifthen}                                           %%
  %optionally (for landscape tables embedded in another document): %%
    \usepackage{lscape}     


\usepackage{url}
\def\UrlBreaks{\do\/\do-}


%\usepackage{stmaryrd}


%\usepackage{wasysym}
%\newcounter{MYtempeqncnt}
\DeclareMathOperator*{\Res}{Res}
%\renewcommand{\baselinestretch}{2}
\renewcommand\thesection{\arabic{section}}
\renewcommand\thesubsection{\thesection.\arabic{subsection}}
\renewcommand\thesubsubsection{\thesubsection.\arabic{subsubsection}}

\renewcommand\thesectiondis{\arabic{section}}
\renewcommand\thesubsectiondis{\thesectiondis.\arabic{subsection}}
\renewcommand\thesubsubsectiondis{\thesubsectiondis.\arabic{subsubsection}}

% correct bad hyphenation here
\hyphenation{op-tical net-works semi-conduc-tor}

%\lstset{
%language=C,
%frame=single, 
%breaklines=true
%}

%\lstset{
	%%basicstyle=\small\ttfamily\bfseries,
	%%numberstyle=\small\ttfamily,
	%language=Octave,
	%backgroundcolor=\color{white},
	%%frame=single,
	%%keywordstyle=\bfseries,
	%%breaklines=true,
	%%showstringspaces=false,
	%%xleftmargin=-10mm,
	%%aboveskip=-1mm,
	%%belowskip=0mm
%}

%\surroundwithmdframed[width=\columnwidth]{lstlisting}
\def\inputGnumericTable{}                                 %%
\lstset{
%language=C,
frame=single, 
breaklines=true,
columns=fullflexible
}
 

\begin{document}
%
\tikzstyle{block} = [rectangle, draw,
    text width=3em, text centered, minimum height=3em]
\tikzstyle{sum} = [draw, circle, node distance=3cm]
\tikzstyle{input} = [coordinate]
\tikzstyle{output} = [coordinate]
\tikzstyle{pinstyle} = [pin edge={to-,thin,black}]

\theoremstyle{definition}
\newtheorem{theorem}{Theorem}[section]
\newtheorem{problem}{Problem}
\newtheorem{proposition}{Proposition}[section]
\newtheorem{lemma}{Lemma}[section]
\newtheorem{corollary}[theorem]{Corollary}
\newtheorem{example}{Example}[section]
\newtheorem{definition}{Definition}[section]
%\newtheorem{algorithm}{Algorithm}[section]
%\newtheorem{cor}{Corollary}
\newcommand{\BEQA}{\begin{eqnarray}}
\newcommand{\EEQA}{\end{eqnarray}}
\newcommand{\define}{\stackrel{\triangle}{=}}
\bibliographystyle{IEEEtran}
%\bibliographystyle{ieeetr}
\providecommand{\nCr}[2]{\,^{#1}C_{#2}} % nCr
\providecommand{\nPr}[2]{\,^{#1}P_{#2}} % nPr
\providecommand{\mbf}{\mathbf}
\providecommand{\pr}[1]{\ensuremath{\Pr\left(#1\right)}}
\providecommand{\qfunc}[1]{\ensuremath{Q\left(#1\right)}}
\providecommand{\sbrak}[1]{\ensuremath{{}\left[#1\right]}}
\providecommand{\lsbrak}[1]{\ensuremath{{}\left[#1\right.}}
\providecommand{\rsbrak}[1]{\ensuremath{{}\left.#1\right]}}
\providecommand{\brak}[1]{\ensuremath{\left(#1\right)}}
\providecommand{\lbrak}[1]{\ensuremath{\left(#1\right.}}
\providecommand{\rbrak}[1]{\ensuremath{\left.#1\right)}}
\providecommand{\cbrak}[1]{\ensuremath{\left\{#1\right\}}}
\providecommand{\lcbrak}[1]{\ensuremath{\left\{#1\right.}}
\providecommand{\rcbrak}[1]{\ensuremath{\left.#1\right\}}}
\theoremstyle{remark}
\newtheorem{rem}{Remark}
\newcommand{\sgn}{\mathop{\mathrm{sgn}}}
%\providecommand{\abs}[1]{\left\vert#1\right\vert}
%\providecommand{\res}[1]{\Res\displaylimits_{#1}} 
%\providecommand{\norm}[1]{\left\Vert#1\right\Vert}
%\providecommand{\mtx}[1]{\mathbf{#1}}
%\providecommand{\mean}[1]{E\left[ #1 \right]}
\providecommand{\fourier}{\overset{\mathcal{F}}{ \rightleftharpoons}}
%\providecommand{\hilbert}{\overset{\mathcal{H}}{ \rightleftharpoons}}
\providecommand{\system}{\overset{\mathcal{H}}{ \longleftrightarrow}}
	%\newcommand{\solution}[2]{\textbf{Solution:}{#1}}
\newcommand{\solution}{\noindent \textbf{Solution: }}
\newcommand{\myvec}[1]{\ensuremath{\begin{pmatrix}#1\end{pmatrix}}}
\providecommand{\dec}[2]{\ensuremath{\overset{#1}{\underset{#2}{\gtrless}}}}
\DeclarePairedDelimiter{\ceil}{\lceil}{\rceil}
%\numberwithin{equation}{section}
%\numberwithin{problem}{subsection}
%\numberwithin{definition}{subsection}
\makeatletter
%\@addtoreset{figure}{section}
\makeatother
\let\StandardTheFigure\thefigure
%\renewcommand{\thefigure}{\theproblem.\arabic{figure}}
\renewcommand{\thefigure}{\thesection}
%\numberwithin{figure}{subsection}
%\numberwithin{equation}{subsection}
%\numberwithin{equation}{section}
%\numberwithin{equation}{problem}
%\numberwithin{problem}{subsection}
\numberwithin{problem}{section}
%%\numberwithin{definition}{subsection}
%\makeatletter
%\@addtoreset{figure}{problem}
%\makeatother
\makeatletter
\@addtoreset{table}{section}
\makeatother
\let\StandardTheFigure\thefigure
\let\StandardTheTable\thetable
\let\vec\mathbf
\numberwithin{equation}{section}
\vspace{3cm}
\title{%Convex Optimization in Python
	\logo{
	Random Numbers
	}
}
\author{Vedant Bhandare (CS21BTECH11007)}
%\title{
%	\logo{Matrix Analysis through Octave}{\begin{center}\includegraphics[scale=.24]{tlc}\end{center}}{}{HAMDSP}
%}
% paper title
% can use linebreaks \\ within to get better formatting as desired
%\title{Matrix Analysis through Octave}
%
%
% author names and IEEE memberships
% note positions of commas and nonbreaking spaces ( ~ ) LaTeX will not break
% a structure at a ~ so this keeps an author's name from being broken across
% two lines.
% use \thanks{} to gain access to the first footnote area
% a separate \thanks must be used for each paragraph as LaTeX2e's \thanks
% was not built to handle multiple paragraphs
%
% <-this % stops a space
%\thanks{J. Doe and J. Doe are with Anonymous University.}% <-this % stops a space
%\thanks{Manuscript received April 19, 2005; revised January 11, 2007.}}

% note the % following the last \IEEEmembership and also \thanks - 
% these prevent an unwanted space from occurring between the last author name
% and the end of the author line. i.e., if you had this:
% 
% \author{....lastname \thanks{...} \thanks{...} }
%                     ^------------^------------^----Do not want these spaces!
%
% a space would be appended to the last name and could cause every name on that
% line to be shifted left slightly. This is one of those "LaTeX things". For
% instance, "\textbf{A} \textbf{B}" will typeset as "A B" not "AB". To get
% "AB" then you have to do: "\textbf{A}\textbf{B}"
% \thanks is no different in this regard, so shield the last } of each \thanks
% that ends a line with a % and do not let a space in before the next \thanks.
% Spaces after \IEEEmembership other than the last one are OK (and needed) as
% you are supposed to have spaces between the names. For what it is worth,
% this is a minor point as most people would not even notice if the said evil
% space somehow managed to creep in.
% The paper headers
%\markboth{Journal of \LaTeX\ Class Files,~Vol.~6, No.~1, January~2007}%
%{Shell \MakeLowercase{\textit{et al.}}: Bare Demo of IEEEtran.cls for Journals}
% The only time the second header will appear is for the odd numbered pages
% after the title page when using the twoside option.
% 
% *** Note that you probably will NOT want to include the author's ***
% *** name in the headers of peer review papers.                   ***
% You can use \ifCLASSOPTIONpeerreview for conditional compilation here if
% you desire.
% If you want to put a publisher's ID mark on the page you can do it like
% this:
%\IEEEpubid{0000--0000/00\$00.00~\copyright~2007 IEEE}
% Remember, if you use this you must call \IEEEpubidadjcol in the second
% column for its text to clear the IEEEpubid mark.
% make the title area
\maketitle
%\tableofcontents
%\bigskip
\renewcommand{\thefigure}{\theenumi}
\renewcommand{\thetable}{\theenumi}
\begin{abstract}
This solution manual provides solutions and link to codes used for generation of random numbers.
\end{abstract}
%template ends here
\section{Uniform Random Variables}
\begin{enumerate}[label=\thesection.\arabic*
,ref=\thesection.\theenumi]
\item Generate $10^6$ samples of $U$ using a C program and save into a file called uni.dat .
\\
\solution\\ 
Download the following C code and run it to generate samples of U.
\begin{lstlisting}
wget https://github.com/TYCN129/AI1110-Assignments/blob/main/Manual%201/1.1/1.1.c
\end{lstlisting}
\item Load the uni.dat file into python and plot the empirical CDF of $U$ using the samples in uni.dat. The CDF is defined as
\begin{align}
F_{U}(x) = \pr{U \le x}
\end{align}
\\
\solution
Code used to plot empirical CDF of $U$
\begin{lstlisting}
wget https://github.com/TYCN129/AI1110-Assignments/blob/main/Manual%201/1.2/1.2.py
\end{lstlisting}
\begin{figure}[H]
    \includegraphics[width=\columnwidth]{Figure_1.2.png}
    \caption{CDF of $U$}
    \label{fig:my_label}
\end{figure}
\item Find a  theoretical expression for $F_{U}(x)$.\\
\solution\\
\begin{align}
    \displaystyle F_U(x) = \begin{cases} 
    0 & \text{$x < 0$} \\  
    x & \text{$0 \leq x \leq 1$} \\  
    1 & \text{$x > 1$}  
    \end{cases}
\end{align}
\begin{figure}[h!]
    \centering
    \includegraphics[width=\columnwidth]{Fig.png}
    \caption{CDF of $U$}
    \label{fig:my_label}
\end{figure}
\begin{lstlisting}
wget https://github.com/TYCN129/AI1110-Assignments/blob/main/Manual%201/1.3/1.3.py
\end{lstlisting}
\item Find mean and variance of $U$.\\
\solution\\
Mean of Random Variable $U$ is given by,
\begin{equation}
E\sbrak{U} = \frac{1}{N}\sum_{i=1}^{N}U_i
\end{equation}
and its Variance is given by,
\begin{equation}
E\sbrak{\sbrak{U - E\sbrak{U}}^2} = E\sbrak{U^2} - \sbrak{E\sbrak{U}}^2
\end{equation}
Using the above two formulas, we get,\\
\begin{align}
\text{Mean of } U = 0.500031\\
\text{Variance of } U = 0.083247
\end{align}
Download and run the C code for mean and variance
\begin{lstlisting}
wget https://github.com/TYCN129/AI1110-Assignments/blob/main/Manual%201/1.4/1.4.c
\end{lstlisting}
\item Verify your result theoretically that
\begin{equation}
E\sbrak{U^k} = \int_{-\infty}^{\infty}x^kdF_{U}(x)
\end{equation}
\solution\\
For $k = 1$,
\begin{align}
E\sbrak{U} &= \int_{-\infty}^{\infty}xdF_{U}(x)\\
E\sbrak{U} &= \int_{0}^{1}xd(x)\\
E\sbrak{U} &= 0.5
\end{align}
Similarly, for $k = 2$,
\begin{align}
E\sbrak{U^2} &= 0.3333\\
\text{Variance} &= 0.0833
\end{align}
Thus the simulated and theoretical values of mean and variance of $U$ are approximately equal.
\end{enumerate}

\section{Central Limit Theorem}
\begin{enumerate}[label=\thesection.\arabic*
,ref=\thesection.\theenumi]
\item \solution\\
Download the following C code and run it to generate samples of $X$.
\begin{lstlisting}
wget https://github.com/TYCN129/AI1110-Assignments/blob/main/Manual%201/2.1/2.1.c
\end{lstlisting}
\item \solution
\begin{figure}[H]
    \centering
    \includegraphics[width=\columnwidth]{Figure_2.png}
    \caption{CDF of $X$}
    \label{fig:my_label}
\end{figure}
The plot was generated by running the following Python code
\begin{lstlisting}
wget https://github.com/TYCN129/AI1110-Assignments/blob/main/Manual%201/2.2./2.2.py
\end{lstlisting}
The CDF is a non-decreasing function with its range between 0 and 1. It will be continuous if PDF is finite.\\
\item {Load gau.dat in python and plot the empirical PDF of $X$ using the samples in gau.dat. The PDF of $X$ is defined as
\begin{align}
p_{X}\brak{x} = \frac{d}{dx}F_{X}\brak{x}
\label{eq:pdf_cdf}
\end{align}
What properties does the PDF have?}\\
\solution \\
The empirical PDF of $X$ is plotted by the Python code
\begin{lstlisting}
wget https://github.com/TYCN129/AI1110-Assignments/blob/main/Manual%201/2.3/2.3.py
\end{lstlisting}
The PDF takes non-negative values and area under its curve is 1.
\clearpage
\begin{figure}[h!]
    \centering
    \includegraphics[width=\columnwidth]{a.png}
    \caption{The empirical PDF of $X$}
    \label{fig:my_label}
\end{figure}

\item Find the mean and variance of $X$ by writing a C program.
\\
\solution\\
Run the following C file
\begin{lstlisting}
wget https://github.com/TYCN129/AI1110-Assignments/blob/main/Manual%201/2.4/2.4.c
\end{lstlisting}
\begin{align}
E\sbrak{X}  &= 0.000630 \\
\text{var}\sbrak{X}  &= 1.000149
\end{align}

\item Given that 
\begin{align}
p_{X}\brak{x} = \frac{1}{\sqrt{2\pi}}\exp\brak{-\frac{x^2}{2}}, -\infty < x < \infty,
\end{align}
repeat the above exercise theoretically.\\
\solution\\
\begin{align}
    E\sbrak{X} &= \int_{-\infty}^{\infty}x p_{X}\brak{x} dx \\
    E\sbrak{X} &= \int_{-\infty}^{\infty}x
    \frac{1}{\sqrt{2\pi}}\exp\brak{-\frac{x^2}{2}} dx    
\end{align}
Taking $\frac{x^2}{2} = t$,
\begin{align}
    E\sbrak{X} &= -\int_{\infty}^{\infty}
    \frac{1}{\sqrt{2\pi}}\exp\brak{-t} dt\\
    E\sbrak{X} &= 0
\end{align}
To calculate variance,
\begin{align}
    var\sbrak{X} &= E\sbrak{(X - E\sbrak{X})^2}\\
    var\sbrak{X} &= E\sbrak{X^2}\\
    var\sbrak{X} &= \int_{-\infty}^{\infty}x^2 p_{X}\brak{x} dx \\
    var\sbrak{X} &= \int_{-\infty}^{\infty}x^2
    \frac{1}{\sqrt{2\pi}}\exp\brak{-\frac{x^2}{2}} dx 
\end{align}
We know that,
\begin{align}
    \int_{-\infty}^{\infty}x^2\exp\brak{-\frac{x^2}{2}} &= \sqrt{2\pi}\\
    var\sbrak{X} &= 1
\end{align}
\begin{figure}[h!]
    \centering
    \includegraphics[width=\columnwidth]{Figure.png}
    \caption{PDF of $X$}
    \label{fig:my_label}
\end{figure}

\begin{figure}[h!]
    \centering
    \includegraphics[width=\columnwidth]{Figure_2.5_2.png}
    \caption{CDF of $X$}
    \label{fig:my_label}
\end{figure}
\clearpage
The Python codes plot the CDF and PDF of $X$
\begin{lstlisting}
wget https://github.com/TYCN129/AI1110-Assignments/blob/main/Manual%201/2.5/2.5.py

wget https://github.com/TYCN129/AI1110-Assignments/blob/main/Manual%201/2.5/2.5_2.py
\end{lstlisting}
\end{enumerate}

\section{From Uniform to Other}
\begin{enumerate}[label=\thesection.\arabic*
,ref=\thesection.\theenumi]
    \item Generate samples of
\begin{align}
V = -2\ln\brak{1-U}
\end{align}
and plot its CDF. \\
\solution\\
Download and run the following code to generate samples of $V$
\begin{lstlisting}
wget https://github.com/TYCN129/AI1110-Assignments/blob/main/Manual%201/3.1/3.1.c
\end{lstlisting}
The following Python code plots CDF of $V$
\begin{lstlisting}
wget https://github.com/TYCN129/AI1110-Assignments/blob/main/Manual%201/3.1/3.1.py
\end{lstlisting}
\begin{figure}[h!]
    \centering
    \includegraphics[width=\columnwidth]{Figure_3.1.png}
    \caption{The empirical CDF of $V$}
    \label{fig:my_label}
\end{figure}

\item Find a theoretical expression for $F_V\brak{x}$.
\\
\solution 
\begin{align}
	F_V\brak{x} &= Pr\brak{V\le x} \\
	F_V\brak{x} &= Pr\brak{-2\ln{\brak{1-U}} \le x}	\\
	F_V\brak{x} &= Pr\brak{\ln{\brak{1-U}} \ge -\frac{x}{2}}	\\
	F_V\brak{x} &= Pr\brak{1-U \ge \exp{\brak{-\frac{x}{2}} }}	\\
	F_V\brak{x} &= Pr\brak{U \le 1- \exp{\brak{-\frac{x}{2}} }} \\
	F_V\brak{x} &= F_U\brak{1- \exp{\brak{-\frac{x}{2}} }}		
\end{align}
\begin{small}
\begin{align}
	F_V\brak{x} &=
    \begin{cases}  
        0 & 1- \exp{\brak{-\frac{x}{2}} } < 0\\
        1- \exp{\brak{-\frac{x}{2}} } & 0 \le 1- \exp{\brak{-\frac{x}{2}} } \le 1 \\
        1 & 1 - \exp{\brak{-\frac{x}{2}} > 1 }
    \end{cases}
\end{align}
\end{small}
This simplifies to
\begin{align}
	F_V\brak{x} &=
    \begin{cases}  
        0 & x < 0\\
        1- \exp{\brak{-\frac{x}{2}}} & x \ge 0
    \end{cases}
    \label{eq:F_V}
\end{align}
The following python code plots the theoritical CDF
\begin{lstlisting}
wget https://github.com/TYCN129/AI1110-Assignments/blob/main/Manual%201/3.2/3.2.py
\end{lstlisting}
\begin{figure}[H]
    \centering
    \includegraphics[width=\columnwidth]{Figure_3.2.png}
    \caption{CDF of $V$}
    \label{fig:my_label}
\end{figure}
\end{enumerate}

\section{Triangular Distribution}
\begin{enumerate}[label=\thesection.\arabic*
,ref=\thesection.\theenumi]
    \item Generate
    \begin{align}
        T = U_1 + U_2
    \end{align}

\solution\\
Download and run the following C code to generate tri.dat file.
\begin{lstlisting}
wget https://github.com/TYCN129/AI1110-Assignments/blob/main/Manual%201/4.1/4.1.c
\end{lstlisting}
\item Find the CDF of $T$.\\
\solution
\begin{figure}[h!]
    \centering
    \includegraphics[width=\columnwidth]{Figure_4.2.png}
    \caption{CDF of $T$}
    \label{fig:my_label}
\end{figure}
\clearpage
The following code plots the CDF of $T$
\begin{lstlisting}
wget https://github.com/TYCN129/AI1110-Assignments/blob/main/Manual%201/4.2/4.2.py
\end{lstlisting}
\\
\item Find the CDF of $T$.\\
\solution
\begin{figure}[H]
    \centering
    \includegraphics[width=\columnwidth]{Figure_4.3.png}
    \caption{PDF of $T$}
    \label{fig:my_label}
\end{figure}
The following code plots the PDF of $T$
\begin{lstlisting}
wget https://github.com/TYCN129/AI1110-Assignments/blob/main/Manual%201/4.3/4.3.py
\end{lstlisting}

\item Find the theoritical expressions for PDF and CDF of $T$.\\
\solution\\
\begin{align}
    p_T(x) &= p_{U_1 + U_2}(x) = p_{U_1}(x) * p_{U_2}(x)\\
    p_T(x) &= \int_{-\infty}^{\infty}p_{U_1}(\tau)p_{U_2}(x - \tau)\\
    p_T(x) &= \int_0^1p_{U_2}(x - \tau)
\end{align}
\begin{align}
    \displaystyle p_T(x) = \begin{cases} 
    0 & \text{$x \leq 0$} \\  
    \int_0^x 1d\tau & \text{$0 < x < 1$} \\  
    \int_{x - 1}^1 1d\tau & \text{$1 \leq x < 2$} \\
    0 & \text{$x > 2$}
    \end{cases}
\end{align}
\begin{align}
    \displaystyle p_T(x) = \begin{cases} 
    0 & \text{$x \leq 0$} \\  
    x & \text{$0 < x < 1$} \\  
    2 - x & \text{$1 \leq x < 2$} \\
    0 & \text{$x > 2$}
    \end{cases}
\end{align}
Expression for CDF can be obtained by integrating $p_T(x)$ w.r.t. $X$
\begin{align}
    \displaystyle F_T(x) = \begin{cases} 
    0 & \text{$x \leq 0$} \\  
    \frac{x^2}{2} & \text{$0 < x < 1$} \\  
    -\frac{x^2}{2} + 2x - 1 & \text{$1 \leq x < 2$} \\
    1 & \text{$x > 2$}
    \end{cases}
\end{align}

\item Verify the results through a plot.\\
\solution\\
\begin{figure}[h!]
    \centering
    \includegraphics[width=\columnwidth]{Figure_4.5p.png}
    \caption{Theoretical PDF of $T$}
    \label{fig:my_label}
\end{figure}
\begin{figure}[H]
    \centering
    \includegraphics[width=\columnwidth]{Figure_4.5c.png}
    \caption{The CDF of $T$}
    \label{fig:my_label}
\end{figure}
PDF and CDF plotted by the Python codes
\begin{lstlisting}
wget https://github.com/TYCN129/AI1110-Assignments/blob/main/Manual%201/4.5/4.5_1.py

wget https://github.com/TYCN129/AI1110-Assignments/blob/main/Manual%201/4.5/4.5_1.py
\end{lstlisting}
\end{enumerate}
\end{document}
